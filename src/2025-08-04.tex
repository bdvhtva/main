\documentclass[12pt]{article}
\usepackage{amsmath}
\usepackage{mathptmx}
\usepackage[top=1.5cm,bottom=2cm,left=2cm,right=2cm]{geometry}
\usepackage[utf8]{vietnam}
\begin{document}
\begin{center}
    \begin{minipage}[t]{0.45\textwidth}
        \centering
        BỒI DUỠNG VĂN HÓA TOÁN VĂN ANH \\
        QUẬN TÂN BÌNH \\
        \textbf{\underline{ĐỀ THI THỬ}} \\
        \textit{(Không sử dụng máy tính)}
    \end{minipage}
    \hspace{0.08\textwidth}
    \begin{minipage}[t]{0.45\textwidth}
        \centering
        \textbf{ĐỀ KIỂM TRA TOÁN 9 – HỌC KỲ I}\\
        \textbf{Môn thi: TOÁN} \\
        \textbf{Ngày thi: 04 tháng 8 năm 2025} \\
        \textbf{Thời gian làm bài: 90 phút} \\
    \end{minipage}
\end{center}
\bigskip
\begin{center}
    \begin{minipage}{0.9\textwidth}
        \textbf{Bài 1. (2 điểm)} Giải phương trình tích
        \begin{itemize}
            \item[a)] $(x - 3)(x + 5) = 0$
            \item[b)] $(2x - 1)(x + 2)(x - 4) = 0$
            \item[c)] $(x^2 - 9)(x + 1) = 0$
            \item[d)] $(x^2 - 5x)(x + 4) = 0$
        \end{itemize}
        \textbf{Bài 2. (3 điểm)} Giải các hệ phương trình bậc nhất hai ẩn
        \begin{itemize}
            \item[a)]
                  $\begin{cases}
                          2x + 3y = 12 \\
                          x - y = 1
                      \end{cases}$
            \item[b)]
                  $\begin{cases}
                          3x - 2y = 7 \\
                          4x + y = 9
                      \end{cases}$
            \item[c)] Tổng của hai số là 25. Nếu gấp số thứ nhất lên 3 lần và giảm số thứ hai đi 5 đơn vị thì tổng mới bằng 50. Tìm hai số đó.
            \item[d)]  Một đội công nhân dự định hoàn thành một công trình trong 15 ngày. Nhưng sau 6 ngày thì được tăng thêm 5 người, nhờ đó công trình hoàn thành sớm hơn 3 ngày so với dự định. Biết năng suất làm việc của mỗi người là như nhau. Hỏi ban đầu đội có bao nhiêu người?
        \end{itemize}
        \textbf{Bài 3. (2 điểm)} Bài toán thực tế – hệ phương trình \\
        Một cửa hàng bán hai loại áo: loại A giá 120.000 đồng/chiếc và loại B giá 100.000 đồng/chiếc. Trong một ngày, cửa hàng bán được tổng cộng 30 chiếc, thu được 3.300.000 đồng. Hỏi cửa hàng đã bán bao nhiêu chiếc mỗi loại? \\
        \\
        \textbf{Bài 4. (2 điểm)} Bài toán chuyển động \\
        Một người đi xe máy từ thành phố A đến thành phố B cách nhau 90 km. Nửa quãng đường đầu, người đó đi với vận tốc 30 km/h. Nửa quãng đường sau, người đó đi với vận tốc 45 km/h. Hỏi người đó đi hết quãng đường trong bao lâu? \\
        \\
        \textbf{Bài 5. (1 điểm)} Bài toán tỉ lệ phần trăm \\
        Một cửa hàng dự định bán một lô hàng với giá niêm yết, tổng thu về sẽ là 6.000.000 đồng. Tuy nhiên, cửa hàng quyết định giảm giá 20\% cho mỗi sản phẩm. Nhờ đó, số lượng sản phẩm bán ra tăng thêm và tổng thu nhập sau khi giảm giá là 5.400.000 đồng. Hỏi số lượng sản phẩm bán ra đã tăng bao nhiêu phần trăm so với ban đầu?
    \end{minipage}
\end{center}
\end{document}
