\documentclass[12pt]{article}
\usepackage[utf8]{inputenc}
\usepackage{enumitem}
\setlist[itemize]{leftmargin=\parindent, itemsep=0pt, parsep=0pt, topsep=0pt}
\usepackage[T5]{fontenc}
\usepackage[vietnamese]{babel}
\usepackage[margin=2.5cm]{geometry}
\usepackage{mathptmx}
\usepackage{fancyhdr}
\usepackage{tikz}
\tikzset{every path/.style={draw=cyan, very thick}}
\usetikzlibrary{calc,positioning,arrows.meta}
\usepackage{caption}
\usepackage{paracol}
\columnratio{0.55,0.43}
\setlength{\columnsep}{0.02\textwidth}
\pagestyle{fancy}
\fancyhf{}
\fancyhead[L]{\textbf{TÊN TRUNG TÂM}}
\fancyhead[C]{\textbf{ZALO LIÊN HỆ}}
\fancyhead[R]{\textbf{Toán 6}}
\fancyfoot[C]{\thepage}
\renewcommand{\footrulewidth}{0.4pt}
\renewcommand\thesection{\arabic{section}.}

\title{%
\textbf{%
\begin{tabular}{ll}
Bài 2 & Hình chữ nhật – Hình thoi \\
      & Hình bình hành – Hình thang cân \\
      & {\small Từ khoá: Hình chữ nhật; Hình thoi; Hình bình hành; Hình thang cân.
} \\ % smaller subtitle
\end{tabular}%
}}
\date{}

\begin{document}
\maketitle
\thispagestyle{fancy}
\section{Hình chữ nhật}
\begin{paracol}{2}
Cho hình chữ nhật \(ABCD\) (Hình 1).
\begin{itemize}
  \item[a)] Đo rồi so sánh các cạnh và góc của hình chữ nhật.
  \item[b)] Hãy kiểm tra xem hai cặp cạnh \(AB\) và \(CD\), \(BC\) và \(AD\) có song song với nhau không.
  \item[c)] \(AC\) và \(BD\) được gọi là hai \textbf{đường chéo} của hình chữ nhật.
\end{itemize}
Hãy đo rồi so sánh \(AC\) và \(BD\).
\switchcolumn
\centering
\begin{tikzpicture}[scale=1]
  \def\w{3.5}
  \def\h{2}
  \def\ra{0.25}
  \coordinate (A) at (0,\h);
  \coordinate (B) at (\w,\h);
  \coordinate (C) at (\w,0);
  \coordinate (D) at (0,0);
  \coordinate (O) at ($ (A)!0.5!(C) $);
  \draw (A)--(B)--(C)--(D)--cycle;
  \draw (A)--(C);
  \draw (B)--(D);
  \node[below] at (O) {$O$};
  \begin{scope}[shift={(A)}] \draw (0,0)--(\ra,0)--(\ra,-\ra)--(0,-\ra); \end{scope}
  \begin{scope}[shift={(B)}] \draw (0,0)--(0,-\ra)--(-\ra,-\ra)--(-\ra,0); \end{scope}
  \begin{scope}[shift={(C)}] \draw (0,0)--(-\ra,0)--(-\ra,\ra)--(0,\ra); \end{scope}
  \begin{scope}[shift={(D)}] \draw (0,0)--(\ra,0)--(\ra,\ra)--(0,\ra); \end{scope}
  \node[above left]  at (A) {A};
  \node[above right] at (B) {B};
  \node[below right] at (C) {C};
  \node[below left]  at (D) {D};
  \node[inner sep=0pt, outer sep=0pt] at ($(D)!0.5!(C) + (0,-1)$) {Hình 1};
\end{tikzpicture}
\vspace{-\baselineskip}
\end{paracol}
\section{Hình thoi}
\begin{paracol}{2}
Cho hình thoi \(ABCD\) như Hình 2.
\begin{itemize}
  \item[a)] Hãy đo rồi so sánh các cạnh của hình thoi.
  \item[b)] Hãy kiểm tra xem hai cặp cạnh \(AB\) và \(CD\), \(BC\) và \(AD\) có song song với nhau không.
  \item[c)] \(AC\) và \(BD\) được gọi là hai \textbf{đường chéo} của hình thoi. Dùng êke để kiểm tra xem hai đuờng chéo có vuông góc với nhau hay không.
\end{itemize}
\switchcolumn
\centering
\begin{tikzpicture}[scale=1]
  \def\hx{1.73} 
  \def\hy{1}
  \def\ra{0.18} 
  \coordinate (O) at (0,0);
  \coordinate (A) at (-\hx,0);
  \coordinate (B) at (0,\hy);
  \coordinate (C) at (\hx,0);
  \coordinate (D) at (0,-\hy); 
  \draw (A)--(B)--(C)--(D)--cycle;
  \draw (A)--(C);
  \draw (B)--(D);
  \begin{scope}[shift={(O)}]
    \draw (0,0)--(\ra,0)--(\ra,\ra)--(0,\ra);
  \end{scope}
  \node[below right] at (O) {$O$};
  \node[left]  at (A) {A};
  \node[above] at (B) {B};
  \node[right] at (C) {C};
  \node[below] at (D) {D};
  \node at ($(D) + (0,-1)$) {Hình 2};
\end{tikzpicture}
\vspace{-\baselineskip}
\end{paracol}
\section{Hình bình hành}
\begin{paracol}{2}
Cho hình bình hành \(ABCD\) như Hình 3.
\begin{itemize}
  \item[a)] Hãy đo rồi so sánh cạnh \(AB\) và \(CD\); cạnh \(BC\) và \(AD\).
  \item[b)] Hãy kiểm tra xem hai cặp cạnh \(AB\) và \(CD\), \(BC\) và \(AD\) có song song với nhau không.
  \item[c)] \(AC\) và \(BD\) được gọi là hai \textbf{đường chéo} của hình bình hành.
\end{itemize}
Hai đường chéo \(AC\) và \(BD\) cắt nhau tại \(O\). Hãy so sánh \(OA\) và \(OC\); \(OB\) và \(OD\).
\switchcolumn
\centering
\begin{tikzpicture}[scale=1]
  \def\w{3.5}
  \def\h{2}
  \coordinate (D) at (0,0);
  \coordinate (C) at (\w,0);
  \coordinate (A) at (1.0,\h);
  \coordinate (B) at (\w+1.0,\h); 
  \draw (A)--(B)--(C)--(D)--cycle;
  \draw (A)--(C);
  \draw (B)--(D); 
  \coordinate (O) at ($ (A)!0.5!(C) $);
  \node[below] at (O) {$O$}; 
  \node[above left]  at (A) {A};
  \node[above right] at (B) {B};
  \node[below right] at (C) {C};
  \node[below left]  at (D) {D};
  \node at ($(D)!0.65!(C) + (0,-1)$) {Hình 3};
\end{tikzpicture}
\vspace{-\baselineskip}
\end{paracol}
\section{Hình thang cân}
\begin{paracol}{2}
Cho hình thang \(ABCD\) như Hình 4.
\begin{itemize}
  \item[a)] Hãy đo rồi so sánh hai cạnh bên \(BC\) và \(AD\).
  \item[b)] Hãy kiểm tra xem \(AB\) có song song với \(CD\) hay không.
  \item[c)] \(AC\) và \(BD\) được gọi là hai \textbf{đường chéo}. Hãy đo rồi so sánh \(AC\) và \(BD\).
\end{itemize}
\switchcolumn
\begin{center}
\begin{tikzpicture}[scale=1] 
  \def\topw{2.5}
  \def\botw{3.5} 
  \def\h{2}
  \pgfmathsetmacro{\offset}{(\botw - \topw) / 2}
  \coordinate (D) at (0,0);
  \coordinate (C) at (\botw,0);
  \coordinate (B) at (\offset+\topw,\h);
  \coordinate (A) at (\offset,\h);
  \draw (A)--(B)--(C)--(D)--cycle;
  \draw (A)--(C);
  \draw (B)--(D);
  \coordinate (O) at ($ (A)!0.4!(C) $);
  \node[below] at (O) {$O$}; 
  \node[above left]  at (A) {A};
  \node[above right] at (B) {B};
  \node[below right] at (C) {C};
  \node[below left]  at (D) {D};
  \node at ($(D)!0.5!(C) + (0,-1)$) {Hình 4};
\end{tikzpicture}
\end{center}
\end{paracol}
\end{document}
